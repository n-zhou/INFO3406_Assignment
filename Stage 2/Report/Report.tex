\documentclass[font=10pt]{article}
\usepackage[left=3cm,right=3cm,top=3cm,bottom=3cm]{geometry}
\usepackage{graphicx}
\usepackage{float}
%\setcounter{tocdepth}{4}
\graphicspath{ {./images/} }
\begin{document}

  \begin{titlepage}
    \centering
    \title{\textbf{INFO3406 Assignment Stage 2}}
    \author{
      Nick Zhou 460363707\\
      Linzi Zhu 46xxxxxxx
    }
    \date{October 2018}
    \maketitle
    \includegraphics[width=5cm]{usyd}
  \end{titlepage}

  \begin{tableofcontents}
    \tableofcontents
  \end{tableofcontents}

  \section{Section 1: Setup}
    \subsection{Hypothesis}
    % State your research question(s). State null and alternative hypotheses
    \underline{Hypotheses}
    \begin{itemize}
      \item H0 (the null Hypotheses): Lexicon usage in movie titles \textit{has no effect} on the performance of a movie.
      \item H1 (the alternative Hypotheses): Lexicon usage in movie does \textit{has an effect} on the performance of a movie.
    \end{itemize}

    \subsection{Reliability}
    % Describe how you will quantify reliability e.g. significance testing, confidence intervals. If appropriate, describe how you will measure effectiveness e.g. regression r-square, clustering V-measure, classification f1-score.

    \subsection{Dataset}
    % Identify datasets and the data you derived from them

  \section{Section 2: Approach}
  Approach...
  \section{Section 3: Results}
  The results to appear to indicate that...
  \section{Conclusion}
  There was no evidence to support the alternative hypothesis and as a result we retain the null hypothesis\
  which is that there is effect on movie performance based on their lexicon usage in titles.
\end{document}
