\documentclass[font=10pt]{article}
\usepackage[left=3cm,right=3cm,top=3cm,bottom=3cm]{geometry}
\usepackage{graphicx}
\usepackage{float}
\graphicspath{ {./images/} }
\begin{document}

  \begin{titlepage}
      \centering
      \title{INF3406 Assignment Stage 2}
      \author{
        Nick Zhou 460363707\\
        Linzi Zhu 46xxxxxxx
      }
      \date{October 2018}
      \maketitle
      \includegraphics[width=5cm]{usyd}
  \end{titlepage}

  \section{Section 1: Setup}
  Hypothesis:
  H0 (the null Hypothesis): Lexicon usage in movie titles does not have any effect on the performance of a movie.
  H1 (the alternative Hypothesis): Lexicon usage in movie does have an effect on the performance of a movie.
  \\\\
  \section{Section 2: Approach}
  Approach...
  \\\\
  \section{Section 3: Results}
  The results to appear to indicate that...
  \\\\
  \section{Conclusion}
  There was no evidence to support the alternative hypothesis and as a result we retain the null hypothesis\
  which is that there is effect on movie performance based on their lexicon usage in titles.
\end{document}
